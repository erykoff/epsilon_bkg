\documentclass{article}

\newcommand{\obs}{\mathrm{obs}}
\newcommand{\aper}{\mathrm{aper}}
\newcommand{\aperin}{\mathrm{aper\_in}}
\newcommand{\aperout}{\mathrm{aper\_out}}
\newcommand{\true}{\mathrm{true}}
\newcommand{\bkgoffset}{\varepsilon}
\newcommand{\ZP}{\mathrm{ZP}}

\begin{document}
\title{Background Offsets from Aperture Fluxes}
\author{Eli S. Rykoff}

\section{Estimating Background Offsets}

What is the effect of an additive background offset on photometric
measurements?  I believe the equations here have been worked out by others
(e.g. Evans et al. 2018), but I have not seen them written explicitly or in
this detail.

Let us assume we have a constant background offset which is small, so we will
call it $\bkgoffset$.  This will be measured either in instrumental units
($\mathrm{ADU}/\mathrm{pixel}^2$) or in physical units
($\mathrm{nJy}/\mathrm{arcsec}^2$).

The observed aperture magnitude including an extra background contribution is:
%
\begin{equation}
  m_\aper^\obs = \ZP - 2.5\log_{10}(f_\aper^\true + \pi r_\aper^2 \bkgoffset),
\end{equation}
%
where $m_\aper^\obs$ is the observed aperture magnitude, $\ZP$ is an arbitrary zeropoint, and
$f_\aper^\true$ is the true aperture flux of the object.  We can split this
into two components:
%
\begin{eqnarray}
  m_\aper^\obs &=& \ZP - 2.5\log_{10}\left [ f_\aper^\true \left ( 1 + \frac{\pi
      r_\aper^2 \bkgoffset}{f_\aper^\true} \right ) \right ]\\
  &=& \ZP - 2.5\log_{10}f_\aper^\true - 2.5\log_{10} \left ( 1 + \frac{\pi
    r_\aper^2 \bkgoffset}{f_\aper^\true} \right ).
\end{eqnarray}
%
Assuming that $\bkgoffset$ is small (specifically, $\frac{\pi r^2
  \bkgoffset}{f} \ll 1$), then we have:
%
\begin{equation}
  m_\aper^\obs \approx \ZP - 2.5\log_{10}f_\aper^\true - \frac{k \pi
    r_\aper^2 \bkgoffset}{f_\aper^\true},
\end{equation}
%
where $k \equiv 2.5/\ln(10)$.

The impact of the background offset is thus a \emph{flux}-dependent offset in
\emph{magnitude}.  This is breaking the assumption that all the variations that
we see (due to system throughput, or ratio of stellar flux with different
apertures) are multiplicative in flux and thus additive in magnitude.

Of course, with a single star or even a set of stars with a single magnitude
measurement we could not isolate the effect of the background offset.  By
\emph{comparing} aperture magnitudes we can.  There are a few implicit
assumptions in what follows.  First of all, we must use apertures that are
large enough to avoid any variations in the core of the psf, which we expect
varies more rapidly spatially than the outer wings of the psf.  Second, any
spatial variations in the wings of the psf should not be correlated with flux
(that is, the bright and faint stars are not segregated by psf size).

Now taking the difference between a two large apertures, which we call
$\aperin$ (with radius $r_{\mathrm{in}}$) and $\aperout$ (with radius
$r_{\mathrm{out}}$):
%
\begin{eqnarray}
  m_\aperin^\obs - m_\aperout^\obs &=& \ZP - 2.5\log_{10}f_\aperin^\true -
  \frac{k \pi r_{\mathrm{in}}^2 \bkgoffset}{f_\aperin^{\true}}\\
  &&- \ZP +
  2.5\log_{10}f_\aperout^\true + \frac{k \pi r_{\mathrm{out}}^2
    \bkgoffset}{f_\aperout^{\true}}
\end{eqnarray}
%
\begin{eqnarray}
  m_\aperin^\obs - m_\aperout^\obs &=& -2.5\log_{10}f_\aperin^\true +
  2.5\log_{10}f_\aperout^\true\\
  && + \frac{k \pi r_{\mathrm{out}}^2 \bkgoffset}{f_\aperout^\true} - \frac{k \pi
    r_{\mathrm{in}}^2 \bkgoffset}{f_\aperin^\true}.
\end{eqnarray}

For a constant PSF, the difference between the inner and outer true flux will
be a simple multiplicative factor, $c$, which we define such that:
\begin{equation}
  f_\aperout^\true \equiv c f_\aperin^\true.
\end{equation}

Splitting this out, we get:
%
\begin{eqnarray}
  m_\aperin^\obs - m_\aperout^\obs &=& -2.5\log_{10}f_\aperin^\true +
  2.5\log_{10}(c f_\aperin^\true)\\
  && + \frac{k \pi r_{\mathrm{out}}^2 \bkgoffset}{c f_\aperin^\true} - \frac{k
    \pi r_{\mathrm{in}}^2 \bkgoffset}{f_\aperin^\true}.
\end{eqnarray}
%
\begin{eqnarray}
  m_\aperin^\obs - m_\aperout^\obs &=& 2.5\log_{10}c\\
  && + \frac{1}{c}\left [ \frac{k \pi r_{\mathrm{out}}^2
      \bkgoffset}{f_\aperin^\true} - \frac{c k \pi r_{\mathrm{in}}^2
      \bkgoffset}{f_\aperin^\true} \right ].
\end{eqnarray}

In the end, this becomes:
\begin{equation}
  m_\aperin^\obs - m_\aperout^\obs = 2.5\log_{10}c + \frac{k \pi \bkgoffset
    (r_{\mathrm{out}}^2 - c r_{\mathrm{in}}^2)}{c} \frac{1}{f_\aperin^\true}.
\end{equation}
%
Thus, for a sample of stars that share a background offset, there is a linear
relation between $m_\aperin^\obs - m_\aperout^\obs$ and $1/f_\aperin^\true$,
where the slope depends on known aperture radii, the unknown $\bkgoffset$, and
the flux ratio $c$.

In practice, we fit the linear relation assuming $c=1.0$, use the y-intercept
to recover $c$ (which for typical apertures used is on the order of $1.01 -
1.02$), and refit fixing the value of $c$.  At the moment we assume that the
effect of using $f_\aperin^\obs$ as an estimate of $f_\aperin^\true$ for the
purposes of the fit, but we could in principle estimate $f_\aperin^\true$ from
the initial fit.


\end{document}
